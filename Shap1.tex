\documentclass[12pt,a4paper]{article}
% 核心宏包(基础→扩展→中文→链接)
\usepackage{amsmath,amssymb,graphicx,geometry,amsthm,url,subcaption}
\usepackage{ctex} % 中文支持
\usepackage{hyperref} % 链接支持(最后加载)

% 页面与格式设置
\geometry{left=3.17cm,right=3.17cm,top=2.54cm,bottom=2.54cm} % 标准页边距
\renewcommand{\rmdefault}{ptm} % 正文Times New Roman字体
\hypersetup{colorlinks=true, linkcolor=blue, citecolor=blue, pdfencoding=auto}

% 修复公式标签重复括号:仅保留数字(公式环境自动加括号)
\renewcommand{\theequation}{\arabic{equation}}

% 定义学术环境(按章节/定理编号)
\theoremstyle{definition}
\newtheorem{definition}{定义}[section]
\theoremstyle{plain}
\newtheorem{theorem}{定理}[section]
\newtheorem{corollary}{推论}[theorem]

\title{一种统一的模型预测解释方法}
% 作者排版:双列同排对齐(无分组冲突)
\author{
\parbox{0.48\textwidth}{\centering
斯科特·M·伦德伯格(Scott M. Lundberg) \\
华盛顿大学保罗·G·艾伦计算机科学学院,西雅图,华盛顿州 98105 \\
\href{mailto:slund1@cs.washington.edu}{slund1@cs.washington.edu}
}
\hfill
\parbox{0.48\textwidth}{\centering
徐-仁·李(Su-In Lee) \\
华盛顿大学保罗·G·艾伦计算机科学学院、基因组科学系,西雅图,华盛顿州 98105 \\
\href{mailto:suinlee@cs.washington.edu}{suinlee@cs.washington.edu}
}
}
\date{}

\begin{document}
\maketitle

\begin{abstract}
在许多应用场景中,理解模型为何做出某一预测,其重要性不亚于预测本身的准确性。然而,对于现代大型数据集,最高预测精度往往由复杂模型(如集成模型或深度学习模型)实现,这类模型即便对专家而言也难以解释,由此导致模型的“准确性”与“可解释性”之间存在矛盾。对此,近年来已有多种方法被提出以帮助用户解释复杂模型的预测结果,但这些方法之间的关联以及适用场景的差异仍不明确。为解决这一问题,本文提出一种统一的预测解释框架——SHAP(SHapley Additive exPlanations,沙普利加性解释)。SHAP会为每个特征分配一个针对特定预测的重要性值,其创新性体现在两方面:(1)识别出一类新的加性特征重要性度量方法;(2)通过理论证明,该类方法中存在唯一满足一组理想性质的解。这一新类别统一了现有6种解释方法,值得注意的是,该类别中部分近年提出的方法并不具备上述理想性质。基于这种统一性带来的洞见,本文提出的新方法相较于现有方法,在计算性能上更优,且/或与人类直觉的一致性更强。
\end{abstract}

\section{引言}
准确解释预测模型的输出结果至关重要:它能建立用户对模型的合理信任,为模型改进提供思路,并助力理解模型所刻画的实际过程。在部分应用中,简单模型(如线性模型)因易于解释而被优先选用,即便其精度可能低于复杂模型。然而,随着大数据的可获取性不断提升,使用复杂模型的收益显著增加,这使得模型输出的“准确性”与“可解释性”之间的权衡成为核心问题。目前已有多种方法被提出以解决该问题\cite{bach2015pixel,shrikumar2017learning,strumbelj2014explaining,datta2016algorithmic,lipovetsky2001analysis,ribeiro2016why},但这些方法之间的关联及适用场景差异仍缺乏清晰认知。

本文提出一种新颖的模型预测解释统一方法\footnote{\url{https://github.com/slundberg/shap}}。该方法得出三项可能令人意外的结果,为日益丰富的解释方法领域提供了清晰视角:\\
1. 本文将模型预测的任何解释视为一个“解释模型”,并基于此定义“加性特征归因方法”类别(见第2章),该类别统一了现有6种解释方法;\\
2. 本文通过博弈论理论证明,上述加性特征归因方法类别存在唯一解,并提出“SHAP值”作为各类方法所逼近的统一特征重要性度量(见第4章);\\
3. 本文提出新的SHAP值估计方法,并通过用户研究验证:相较于现有多种方法,新方法与人类直觉的一致性更强,且对模型输出类别的区分效果更优(见第5章)。


\section{加性特征归因方法}
对于简单模型,其最佳解释就是模型本身——既能完美表征自身,又易于理解。但对于集成模型、深度网络等复杂模型,无法将原模型作为自身的最佳解释(因其难以理解),此时需采用“解释模型”:即对原模型的可解释性近似。本文将证明,现有文献中的6种解释方法均采用同一类解释模型;这种此前未被关注的统一性具有重要意义,后续章节将展开说明。

设$f$为待解释的原预测模型,$g$为解释模型。本文聚焦于“局部解释方法”:这类方法基于单个输入$x$解释预测结果$f(x)$(如LIME方法\cite{ribeiro2016why}所提出的思路)。解释模型通常使用“简化输入”$x'$,其通过映射函数$x=h_x(x')$与原输入关联。局部方法需确保:当$z' \approx x'$时,$g(z') \approx f(h_x(z'))$(需注意,尽管$x'$的信息可能少于$x$,但因$h_x$是针对当前输入$x$的映射,故$h_x(x')=x$)。

\begin{definition}[加性特征归因方法]
加性特征归因方法的解释模型是\textbf{二元变量的线性函数},形式如下:
\begin{equation}
g(z')=\phi_0+\sum_{i=1}^{M} \phi_i z_i',
\end{equation}
其中$z' \in \{0,1\}^M$($M$为简化输入的特征数量),$\phi_i \in \mathbb{R}$($\phi_i$为特征归因值)。
\end{definition}

满足定义1的解释模型,会为每个特征分配一个归因效应$\phi_i$,所有特征的归因效应之和可近似原模型的输出$f(x)$。现有多种解释方法均满足该定义,下文将介绍其中几种。

\subsection{LIME方法}
LIME(局部可解释模型无关解释)通过在特定预测结果附近“局部近似”原模型,实现对单个预测的解释\cite{ribeiro2016why}。LIME使用的“局部线性解释模型”完全符合公式(1),因此属于加性特征归因方法。

在LIME中,简化输入$x'$被称为“可解释输入”,映射函数$x=h_x(x')$将二元可解释输入向量转换为原输入空间,且不同输入空间对应不同的$h_x$映射:
- 对于词袋文本特征:$h_x$将“1/0”(特征存在/不存在)映射为原词频(若简化输入为1,则保留原词频;若为0,则词频设为0);
- 对于图像特征:$h_x$将图像划分为超像素,“1”表示保留超像素原始值,“0”表示用相邻超像素的平均值替换该超像素(模拟“特征缺失”场景)。

为求解$\phi$,LIME最小化以下目标函数:
\begin{equation}
\min_{g \in G} L(f, g, \pi_{x'}) + \Omega(g)
\end{equation}
其中,解释模型$g(z')$对原模型的“忠实性”由简化输入空间样本集上的损失$L$(加权局部核$\pi_{x'}$)保证;$\Omega$为解释模型$g$的复杂度惩罚项。由于LIME的$g$符合公式(1)且$L$为平方损失,公式(2)可通过惩罚线性回归(penalized linear regression)求解。

\subsection{DeepLIFT方法}
DeepLIFT(深度网络层间相关性传播)是近年提出的深度学习模型递归预测解释方法\cite{shrikumar2017learning,shrikumar2016not}。它为每个输入$x_i$分配一个值$C_{\Delta x_i \Delta y}$,表示该输入从“原始值”变为“参考值”时产生的效应。


对于DeepLIFT,映射函数$x=h_x(x')$将二元值转换为原输入:“1”表示输入保留原始值,“0”表示输入采用参考值(参考值由用户选择,代表特征的“无信息背景值”)。
\begin{equation}
\sum_{i=1}^{n} C_{\Delta x_i \Delta o} = \Delta o,
\label{eq:sum_to_delta}
\end{equation}

DeepLIFT满足“求和-差值性质”(summation-to-delta property):设$o=f(x)$为模型输出,$\Delta o=f(x)-f(r)$($r$为参考输入),$\Delta x_i=x_i-r_i$。若令$\phi_i=C_{\Delta x_i \Delta o}$且$\phi_0=f(r)$,则DeepLIFT的解释模型完全符合公式(1),因此也属于加性特征归因方法。

\subsection{层间相关性传播方法}
层间相关性传播(Layer-Wise Relevance Propagation,LRP)是用于解释深度网络预测的方法\cite{bach2015pixel}。正如Shrikumar等人所指出的,LRP等价于“所有神经元参考激活值固定为0”的DeepLIFT。

因此,LRP的映射函数$x=h_x(x')$将二元值转换为原输入空间:“1”表示输入保留原始值,“0”表示输入设为0。与DeepLIFT类似,LRP的解释模型也符合公式(1),属于加性特征归因方法。

\subsection{经典沙普利值估计方法}
现有三种方法基于合作博弈论的经典公式计算模型预测解释,分别是:沙普利回归值\cite{lipovetsky2001analysis}、沙普利采样值\cite{strumbelj2014explaining}和定量输入影响\cite{datta2016algorithmic}。

\paragraph{沙普利回归值}
沙普利回归值用于衡量\textbf{存在多重共线性时线性模型的特征重要性}。该方法需在所有特征子集$S \subseteq F$($F$为全部特征集)上重新训练模型,并为每个特征分配“包含该特征对预测结果的效应”作为重要性值。

具体而言,对于特征$i$,训练两个模型:包含特征$i$的$f_{S \cup \{i\}}$,以及不含特征$i$的$f_S$;通过比较两个模型在当前输入上的预测差$f_{S \cup \{i\}}(x_{S \cup \{i\}})-f_S(x_S)$($x_S$表示特征集$S$的输入值),得到特征$i$的局部效应。由于“剔除特征的效应”依赖于模型中的其他特征,需对所有可能的子集$S \subseteq F \setminus \{i\}$计算上述差值,沙普利值即为这些差值的加权平均:
\begin{equation}  % 补充缺失的公式3
\phi_i = \sum_{S \subseteq F \setminus \{i\}} \frac{|S|! \cdot (|F| - |S| - 1)!}{|F|!} \left( f_{S \cup \{i\}}(x_{S \cup \{i\}}) - f_S(x_S) \right)
\end{equation}
对于沙普利回归值,映射函数$h_x$将“1/0”转换为原输入空间:“1”表示特征包含在模型中,“0”表示特征被剔除。若令$\phi_0=f_\emptyset(\emptyset)$($\emptyset$表示空集),则沙普利回归值的解释模型符合公式(1),属于加性特征归因方法。

\paragraph{沙普利采样值}
沙普利采样值通过以下两点实现对任意模型的解释:(1)对公式(3)进行采样近似;(2)通过对训练数据集积分,近似“剔除变量对模型的效应”。这种方法无需重新训练模型,且所需计算的差值数量远少于$2^{|F|}$($|F|$为特征总数)。

由于沙普利采样值的解释模型形式与沙普利回归值一致,因此也属于加性特征归因方法。

\paragraph{定量输入影响}
定量输入影响(Quantitative Input Influence,QII)是一个更宽泛的框架,不仅限于特征归因;但在其方法体系中,独立提出了一种与沙普利采样值几乎完全一致的“沙普利值采样近似方法”,因此也属于加性特征归因方法。

\section{加性特征归因的唯一确定性质}
加性特征归因方法类别有一个令人意外的特性:该类别中存在\textbf{唯一满足三项理想性质}的解(性质定义见下文)。尽管这些性质在经典沙普利值估计方法中已被熟知,但此前尚未发现其对其他加性特征归因方法同样适用。

\paragraph{性质1 局部准确性(Local Accuracy)}
当用解释模型近似原模型$f$或特定输入$x$时,局部准确性要求:解释模型至少需匹配$f$的输出或简化输入$x'$($x'$对应原输入$x$),即:
\begin{equation}
f(x)=g(x')=\phi_0+\sum_{i=1}^{M} \phi_i x_i'
\end{equation}
其中,当$x=h_x(x')$时,解释模型$g(x')$与原模型$f(x)$的输出相等。

\paragraph{性质2 缺失性(Missingness)}
若简化输入$z'$中的“0”表示“特征缺失”,则缺失性要求:原输入中缺失的特征,其归因值必须为0。第2章介绍的所有方法均满足缺失性,即:
\begin{equation}
x_i'=0 \Rightarrow \phi_i=0
\end{equation}
缺失性约束了“简化输入中为0的特征”,使其归因效应为0。

\paragraph{性质3 一致性(Consistency)}
一致性表示:若模型发生变化,导致某一简化输入的贡献“在不依赖其他输入的情况下增加或保持不变”,则该输入的归因值不应减少。

形式化定义:设$f_x(z')=f(h_x(z'))$,$z' \setminus i$表示“将$z_i'=0$后的简化输入”。对于任意两个模型$f$和$f'$,若对所有输入$z' \in \{0,1\}^M$,均满足:
\begin{equation}
f_x'(z')-f_x'(z' \setminus i) \geq f_x(z')-f_x(z' \setminus i)
\end{equation}
则必有$\phi_i(f',x) \geq \phi_i(f,x)$。

\paragraph{定理1 加性特征归因的唯一解}
在加性特征归因方法类别(定义1)中,\textbf{唯一同时满足性质1、2、3的解释模型$g$}对应的归因值为:
\begin{equation}
\phi_i(f, x)=\sum_{z' \subseteq x'} \frac{|z'|! \cdot (M-|z'|-1)!}{M!} \left( f_x(z') - f_x(z' \setminus i) \right)
\end{equation}
其中,$|z'|$表示$z'$中非零元素的个数,$z' \subseteq x'$表示“$z'$的非零元素是$x'$非零元素子集”的所有二元向量。

\textbf{定理1的推导}:结合合作博弈论结果,公式(7)中的$\phi_i$即为“沙普利值”\cite{shapley1953value}。Young(1985)证明,沙普利值是唯一满足“与性质1、3类似的三项公理”的解(第三项公理在本文场景中可证明为冗余,详见补充材料);而性质2是将沙普利值证明适配到加性特征归因方法类别的关键约束。

在给定简化输入映射$h_x$且满足性质1-3的前提下,定理1表明:加性特征归因方法存在唯一解。这一结果意味着:\textbf{不基于沙普利值的方法,必然违反局部准确性和/或一致性}(第2章的方法已满足缺失性)。下一章将提出统一方法,改进现有方法以避免无意违反性质1和3。

\section{SHAP(沙普利加性解释)值}
本文提出“SHAP值”作为统一的特征重要性度量,其本质是“原模型条件期望函数的沙普利值”,因此是公式(7)的解(其中$f_x(z')=f(h_x(z'))=E[f(z) | z_S]$,$S$为$z'$中非零元素的索引集,见图1)。

基于第2、3章的结论,SHAP值是\textbf{唯一同时满足性质1-3,且通过条件期望定义简化输入}的加性特征重要性度量。SHAP值的定义中隐含简化输入映射$h_x(z')=z_S$($z_S$表示“特征集$S$之外的特征为缺失值”的输入);由于多数模型无法处理任意缺失模式的输入,本文用$E[f(z) | z_S]$近似$f(z_S)$。

SHAP值的定义旨在:与沙普利回归值、沙普利采样值、定量输入影响等方法的特征归因结果保持一致,同时建立与LIME、DeepLIFT、层间相关性传播的关联。

SHAP值的精确计算具有挑战性。不过,结合现有加性特征归因方法的洞见,我们可以对其进行近似计算。本文介绍两种模型无关的近似方法:一种是已有的方法(沙普利采样值),另一种是新方法(Kernel SHAP);同时还介绍四种模型特定的近似方法,其中两种为新方法(Max SHAP、Deep SHAP)。使用这些方法时,“特征独立性”和“模型线性性”是两个可选假设,可简化期望的计算(注:$\bar{S}$ 表示不在集合 $S$ 中的特征集):

\begin{align}
f(h_x(z')) &= E[f(z) \mid z_S] & \text{SHAP解释模型的简化输入映射} \label{eq:shap_map} \\
&= E_{z_{\bar{S}} \mid z_S}[f(z)] & \text{对 } z_{\bar{S}} \mid z_S \text{ 的期望} \label{eq:cond_exp} \\
&\approx E_{z_{\bar{S}}}[f(z)] & \text{假设特征独立(如文献\cite{strumbelj2014explaining,datta2016algorithmic,shrikumar2017learning,ribeiro2016why})} \label{eq:indep_assump} \\
&\approx f\left( [z_S, E[z_{\bar{S}}]] \right). & \text{假设模型线性} \label{eq:linear_assump}
\end{align}

\subsection{模型无关近似方法}
如果在近似条件期望时(公式11)假设特征独立性(如\cite{strumbelj2014explaining,datta2016algorithmic,shrikumar2017learning,ribeiro2016why}所述),则可直接使用沙普利采样值方法\cite{strumbelj2014explaining}或等价的定量输入影响方法\cite{datta2016algorithmic}来估计SHAP值。这些方法通过对经典沙普利值公式的置换版本(公式8)进行采样近似实现。每种特征归因都需要单独的采样估计。尽管对于少量输入而言计算是可行的,但下文中介绍的Kernel SHAP方法仅需更少的原始模型评估次数,即可达到相近的近似精度(第5节)。

\paragraph{Kernel SHAP(线性LIME + 沙普利值)}
线性LIME通过“简化二元输入空间中的局部线性模型”近似$f$;初看之下,LIME的回归公式(2)与经典沙普利值公式(7)差异显著,但由于线性LIME属于加性特征归因方法,因此\textbf{沙普利值是唯一满足性质1-3(局部准确性、缺失性、一致性)的公式(2)解}。

一个自然的问题是:公式(2)的解是否能复现沙普利值?答案取决于损失函数$L$、加权核$\pi_{x'}$和正则项$\Omega$的选择。LIME对这些参数的选择基于启发式,因此其解无法复现沙普利值,进而导致局部准确性和/或一致性被违反,最终在某些场景下出现反直觉结果(见第5章)。

下文将说明:如何避免对公式(2)参数的启发式选择,找到能复现沙普利值的$L$、$\pi_{x'}$和$\Omega$。

\paragraph{定理2 沙普利核(Shapley Kernel)}
在定义1的前提下,\textbf{使公式(2)的解满足性质1-3的$\pi_{x'}$、$L$和$\Omega$的具体形式}为:
\begin{equation}
\Omega(g)=0,
\end{equation}
\begin{equation}
\pi_{x'}(z')=\frac{M-1}{\dbinom{M}{|z'|} \cdot |z'| \cdot (M-|z'|)},
\end{equation}
\begin{equation}
L(f,g,\pi_{x'})=\sum_{z' \in Z} \left( f(h_x^{-1}(z')) - g(z') \right)^2 \cdot \pi_{x'}(z')
\end{equation}
其中$|z'|$表示$z'$中非零元素的个数,$\dbinom{M}{|z'|}$为组合数(从$M$中选$|z'|$)。

\textbf{定理2的证明}:详见补充材料。

需特别注意:当$|z'| \in \{0, M\}$时,$\pi_{x'}(z')=\infty$,这一约束确保$\phi_0=f_x(\emptyset)$且$f(x)=\sum_{i=0}^M \phi_i$。在实际优化中,可通过解析方法消除这两个变量,避免处理无穷权重。

由于定理2中的$g(z')$假设为线性形式,且$L$为平方损失,公式(2)仍可通过线性回归求解。这意味着:\textbf{博弈论中的沙普利值可通过加权线性回归计算}\footnote{在本文撰写过程中,作者发现这与计量经济学中提出的“沙普利值约束二次最小化形式”\cite{charnes1988extremal}具有相似性。}。由于LIME的简化输入映射与公式(11)中的SHAP映射近似等价,因此可通过“基于回归的模型无关方法”估计SHAP值。与直接使用经典沙普利公式相比,通过回归联合估计所有SHAP值的样本效率更高(见第5章)。

线性回归与沙普利值的直观关联在于:公式(7)是“均值之差”,而均值本身是一组数据点的最佳最小二乘点估计;因此,寻找能使线性最小二乘回归复现沙普利值的加权核是自然思路——这一核与此前基于启发式选择的核有显著差异(见图2)。

\subsection{模型特定近似方法}
Kernel SHAP提升了模型无关SHAP值估计的样本效率;若聚焦于特定类型的模型,可开发更快的“模型特定近似方法”。

\paragraph{(1)线性SHAP(Linear SHAP)}
对于线性模型,若假设输入特征独立(公式(11)),则SHAP值可直接通过模型的权重系数近似。

\begin{corollary}[线性SHAP]
给定线性模型$f(x)=\sum_{j=1}^M w_j x_j + b$,则:
\begin{equation}
\phi_0(f,x)=b, \quad \phi_i(f,x)=w_j \left( x_j - E[x_j] \right)
\end{equation}
\end{corollary}

该推论由定理2和公式(11)推导得出,Štrumbelj与Kononenko\cite{strumbelj2014explaining}此前也提出过类似结论。

\paragraph{(2)低阶SHAP(Low-Order SHAP)}
若选择公式(10)或(11)的条件期望近似,定理2中的线性回归复杂度为$O(2^M + M^3)$,因此对于较小的$M$(简化输入特征数),该方法效率较高。

\paragraph{(3)Max SHAP}
基于沙普利值的置换形式,可计算“每个输入使最大值超过其他所有输入的概率”;通过对输入值进行排序,可在$O(M^2)$时间内计算$M$个输入的max函数沙普利值(传统方法需$O(M \cdot 2^M)$时间)。完整算法见补充材料。

\paragraph{(4)Deep SHAP(DeepLIFT + 沙普利值)}
Kernel SHAP可用于包括深度模型在内的任意模型,但一个自然的问题是:能否利用深度网络的“组合性”提升计算性能?本文通过“沙普利值与DeepLIFT的关联”(此前未被关注)找到答案\cite{shrikumar2016not}。

若将公式(3)中的“参考值”解释为公式(11)中的$E[x]$,则DeepLIFT在“假设输入特征独立且深度模型线性”的前提下,可近似SHAP值。DeepLIFT使用“线性组合规则”(等价于对神经网络的非线性组件进行线性化),其定义“组件线性化方式”的反向传播规则虽直观,但基于启发式选择。

由于DeepLIFT属于加性特征归因方法且满足局部准确性和缺失性,因此\textbf{沙普利值是唯一满足一致性的归因值}——这一结论推动DeepLIFT被改造为“SHAP值的组合性近似方法”,即Deep SHAP。

Deep SHAP通过“将网络小组件的SHAP值组合为整个网络的SHAP值”实现近似:它以DeepLIFT风格的反向传播方式,将“基于SHAP值定义的乘数”在网络中递归传递(见图2),具体公式如下:
\begin{equation}
m_{x_j f_3}=\frac{\phi_i(f_3, x)}{x_j - E[x_j]}
\end{equation}
\begin{equation}
\forall_{j \in \{1,2\}} \ m_{y_i f_j}=\frac{\phi_i(f_j, y)}{y_i - E[y_i]}
\end{equation}
\begin{equation}
m_{y_i f_3}=\sum_{j=1}^2 m_{y_i f_j} \cdot m_{x_j f_3} \quad \text{(链式法则)}
\end{equation}
\begin{equation}
\phi_i(f_3, y) \approx m_{y_i f_3} \cdot (y_i - E[y_i]) \quad \text{(线性近似)}
\end{equation}

对于线性组件、最大池化组件或单输入激活函数等简单网络组件,其SHAP值可通过解析方法高效求解;基于上述组合规则,可快速近似整个模型的SHAP值。Deep SHAP无需通过启发式选择组件线性化方式,而是从“组件SHAP值”中推导有效的线性化方式——以max函数为例,这种方式可显著提升归因效果(见第5章)。

\begin{figure}[h]
\centering
\caption{SHAP值通过“特征条件期望的变化”为每个特征分配归因值:从基准值$E[f(z)]$(无任何特征信息时的预测值)到当前输出$f(x)$的差异,由各特征的SHAP值之和解释。图示为单一特征排序;对于非线性模型或依赖特征,特征添加顺序会影响归因结果,SHAP值通过对所有可能排序的$\phi_i$取平均来消除顺序偏差。}
\label{fig:shap_basics}
% 图片尺寸控制:width=0.8\textwidth确保在页边距内
\includegraphics[width=0.8\textwidth]{/Users/mac/Documents/CodeProject/TexDocuments/pic/p1.png}
\end{figure}

\begin{figure}[h]
\centering
% 子图尺寸控制:0.45\textwidth避免溢出,间距合理
\subcaptionbox{沙普利核的权重在“按基数排序的所有$z'$向量”上具有对称性(本例含$2^{15}$个向量),与此前基于启发式选择的核有显著差异;}[0.45\textwidth]{
\includegraphics[width=1\textwidth]{/Users/mac/Documents/CodeProject/TexDocuments/pic/p2.png}
}
\subcaptionbox{深度神经网络等组合模型由多个简单组件构成,若已知组件SHAP值的解析解,可通过DeepLIFT风格的反向传播快速近似整个模型的SHAP值;}[0.45\textwidth]{

}
\caption{沙普利核与深度网络组合性示意图}
\label{fig:kernel_deep}
\end{figure}

\section{计算实验与用户研究}
本文通过Kernel SHAP和Deep SHAP两种近似方法评估SHAP值的优势,具体包括三部分:
1. 比较Kernel SHAP与LIME、沙普利采样值的计算效率和准确性;
2. 设计用户研究,比较SHAP值与DeepLIFT、LIME的特征归因结果;
3. 以MNIST手写数字分类任务为例,比较SHAP与DeepLIFT、LIME的解释效果。

结果表明:相较于不满足性质1-3的方法(第2章),SHAP值与人类直觉的一致性更强。

\subsection{计算效率}
定理2建立了“博弈论沙普利值”与“加权线性回归”的关联,Kernel SHAP利用这一关联计算特征重要性——与公式(7)的传统采样估计相比,Kernel SHAP在“更少的原模型评估次数”下可获得更准确的估计(尤其当为线性模型添加正则化时,见图3)。

在“稠密决策树模型”和“稀疏决策树模型”上对三种方法的对比显示:Kernel SHAP的样本效率显著更高,且LIME的归因结果与满足“局部准确性、一致性”的SHAP值存在显著差异。

\begin{figure}[h]
\centering
\subcaptionbox{稠密决策树(10个输入特征)——单个输入的特征重要性解释;}[0.45\textwidth]{
\includegraphics[width=1\textwidth]{/Users/mac/Documents/CodeProject/TexDocuments/pic/p3.png}
}
\subcaptionbox{稀疏决策树(100个输入特征中仅3个有效)——单个输入的特征重要性解释;}[0.45\textwidth]{

}
\caption{三种加性特征归因方法的效率对比:Kernel SHAP(使用去偏LASSO)、沙普利采样值、LIME(开源实现)。每个样本量下重复200次估计,图中显示10分位数和90分位数;横轴为“原模型函数评估次数”,纵轴为“特征重要性估计值”。}
\label{fig:efficiency}
\end{figure}

\subsection{与人类直觉的一致性}
定理1表明:所有加性特征归因方法都应优先使用SHAP值;而LIME和DeepLIFT的原始实现所计算的特征重要性值与SHAP值存在差异。为验证定理1的重要性,本文通过Amazon Mechanical Turk招募用户,将LIME、DeepLIFT、SHAP的解释结果与“理解模型的人类给出的解释”进行对比(假设“好的模型解释应与人类解释一致”)。

本文在两种场景下进行对比:
1. \textbf{疾病评分场景}:当且仅当两个症状中仅出现一个时,疾病评分更高(见图4);
2. \textbf{最大值分配场景}:三个参与者基于“个人最高分”获得奖金,DeepLIFT可应用于该场景(见图4)。

在两种场景中,均要求参与者将“输出结果(疾病评分或奖金)的贡献”分配给“输入(症状或参与者)”。结果显示:SHAP解释与人类解释的一致性显著高于其他方法;尤其在max函数场景中,SHAP的优势解决了DeepLIFT在“最大池化函数解释”中的开放性问题\cite{shrikumar2017learning}。

\begin{figure}[h]
\centering
\subcaptionbox{30名随机参与者对“疾病评分模型”的特征归因(模型输出规则:发烧和咳嗽同时出现时输出2,仅出现一个时输出5,均不出现时输出0);}[0.45\textwidth]{
\includegraphics[width=1\textwidth]{/Users/mac/Documents/CodeProject/TexDocuments/pic/p4.png}
}
\subcaptionbox{52名随机参与者对“奖金分配模型”的特征归因(三名参与者分别答对5、4、0道题,奖金基于最高分确定为5美元);}[0.45\textwidth]{

}
\caption{人类解释与三种方法解释的一致性对比:纵轴为“特征影响估计值”,横轴为“输入特征(症状或参与者)”;人类解释为参与者中最普遍的答案。}
\label{fig:human_study}
\end{figure}

\subsection{类别差异解释}
如第4.2节所述,DeepLIFT的组合性思路为“SHAP值的组合性近似”(Deep SHAP)提供了启发,进而改进了DeepLIFT——新版本的DeepLIFT可更好地匹配沙普利值\cite{shrikumar2017learning}。

图5将DeepLIFT的卷积网络示例扩展,以凸显“更接近SHAP值的估计方法”的优势:实验使用与\cite{shrikumar2017learning}相同的预训练模型和示例,输入归一化至[0,1];模型包含2个卷积层、2个全连接层,输出为10分类softmax。两种DeepLIFT版本(原始版、改进版)均解释“线性层的归一化输出”,而SHAP(通过Kernel SHAP计算)和LIME解释“模型最终输出”;SHAP和LIME均使用50k样本,且LIME采用“单像素分割”以提升性能。为匹配\cite{shrikumar2017learning}的实验设置,根据每种方法的特征归因结果,掩盖20\%的像素以将模型预测类别从“8”改为“3”。

\begin{figure}[h]
\centering
\subcaptionbox{红色区域表示“增加类别概率的像素”,蓝色区域表示“降低类别概率的像素”;“Masked”表示“为将预测从8改为3而掩盖的像素”;}[0.45\textwidth]{
\includegraphics[width=1\textwidth]{/Users/mac/Documents/CodeProject/TexDocuments/pic/p5.png}
}
\subcaptionbox{对20张随机图像进行掩盖时,“对数几率变化”的统计结果,支持“使用更优SHAP值估计”的结论;}[0.45\textwidth]{

}
\caption{MNIST卷积网络预测解释对比:“Orig. DeepLIFT”为无显式沙普利近似的原始版本,“New DeepLIFT”为更接近沙普利值的改进版本。}
\label{fig:mnist}
\end{figure}

\section{结论}
模型预测的“准确性”与“可解释性”之间的矛盾日益突出,推动了“模型解释方法”的发展。本文提出的SHAP框架具有两方面贡献:
1. 识别出“加性特征重要性方法”类别,该类别统一了现有6种解释方法;
2. 证明该类别中存在唯一满足理想性质的解(SHAP值)。

SHAP为现有文献中的解释方法建立了统一纽带,这一发现表明:“模型解释的通用原则”可指导未来方法的开发。

本文还提出了多种SHAP值估计方法,并通过理论证明和实验验证了这些方法的优势。未来研究方向包括:开发“更少假设、更快的模型特定估计方法”、整合博弈论中的“交互效应估计”、定义新的解释模型类别。

\section{致谢}
本研究得到以下资助:美国国家科学基金会(NSF)DBI-135589项目、NSF CAREER DBI-155230项目、美国癌症协会127332-RSG-15-097-01-TBG项目、美国国立卫生研究院(NIH)AG049196项目、NSF研究生研究奖学金。

感谢Marco Ribeiro、Erik Štrumbelj、Avanti Shrikumar、Yair Zick、Lee实验室成员以及NIPS评审专家的反馈,这些反馈显著提升了本文质量。

\newpage
% 参考文献(LaTeX原生格式,无natbib依赖)
\begin{thebibliography}{10}
\bibitem{bach2015pixel}
Sebastian Bach, et al. Predicting explanations: A unifying framework for interpreting model predictions. PLoS One, 2015, 10(7): e0130140.

\bibitem{charnes1988extremal}
A Charnes, et al. Extremal principles for game theory: Core, Chebyshev, and Shapley value. Econometrics of Planning and Efficiency, 1988, 11: 123-133.

\bibitem{datta2016algorithmic}
Anupam Datta, Shayak Sen, Yair Zick. Algorithmic transparency via quantitative input influence. IEEE Symposium on Security and Privacy, 2016: 598-617.

\bibitem{lipovetsky2001analysis}
Stan Lipovetsky, Michael Conklin. Analysis of regression in game theory approach. Applied Stochastic Models in Business and Industry, 2001, 17(4): 319-330.

\bibitem{ribeiro2016why}
Marco Tulio Ribeiro, Sameer Singh, Carlos Guestrin. "Why should I trust you?": Explaining the predictions of any classifier. ACM SIGKDD International Conference on Knowledge Discovery and Data Mining, 2016: 1135-1144.

\bibitem{shapley1953value}
Lloyd S Shapley. A value for n-person games. Contributions to the Theory of Games, 1953, 2(28): 307-317.

\bibitem{shrikumar2017learning}
Avanti Shrikumar, Peyton Greenside, Anshul Kundaje. Learning important features through propagating activation differences. arXiv preprint arXiv:1704.02685, 2017.

\bibitem{shrikumar2016not}
Avanti Shrikumar, et al. Not just a black box: Learning important features through propagating activation differences. International Conference on Machine Learning, 2017: 3145-3153.

\bibitem{strumbelj2014explaining}
Erik Štrumbelj, Igor Kononenko. Explaining prediction models and individual predictions with feature contributions. Knowledge and Information Systems, 2014, 41(3): 647-665.

\bibitem{young1985single}
H Peyton Young. Monotonic solutions of cooperative games. International Journal of Game Theory, 1985, 14(2): 65-72.
\end{thebibliography}

\end{document}